%\documentclass[varwidth=true, border=2pt]{standalone}
%\documentclass{article}
\documentclass[
a4paper,                                               % Papierformat
landscape,
oneside,                                               % Einseitig
%twoside,                                              % Zweiseitig
12pt,                                                  % Schriftgröße
pagesize=auto,                                         % schreibt die Papiergröße korrekt ins Ausgabedokument
headsepline,                                           % Linie unter der Kopfzeile
%draft=true                                            % Markiert zu lange und zu kurze Zeilen
]{scrreprt}

\usepackage{bchart}
\usepackage{longtable}
\usepackage{verbatim}
\usepackage[Q=yes]{examplep}
\usepackage{fancyvrb}
\usepackage[utf8]{inputenc}
\usepackage{graphicx}
\usepackage{geometry}
\usepackage{textcomp}

% switching between demo modus and c#
\newcommand{\cs}[2]{#2}
%## print @"\renewcommand{\cs}[2]{#1}"; ##%

%\allowdisplaybreaks
\geometry{a4paper, landscape, left=20mm, right=20mm, top=22mm, bottom=25mm} 

\begin{document}
\textbf{}\hfill{\Huge Test} \hfill \cs{\verb|## print DateTime.Now.ToShortDateString(); ##|}{14.08.2015}

Escape special Characters:\\
%## print Tex.Escape("äöü-_'!\"§$%&/()=?^<>{[]}\\~"); ##%

This is how to make it possible to use a regular Tex-Editor like TeXstudio and have some demo data.\\
\cs{{## print 1; ##}}{Test}

    \begin{bchart}[step=2,max=10]
		%## print o.Value; ##%
		%## if (o.x) { ##%
			\bcbar{3.4}
				\smallskip
		%## } else if (o.y) { ##%
			\bcbar{5.6}
				\medskip
		%## } else if (o.z) { ##%
			\bcbar{7.2}
				\bigskip
		%## } else { ##%
			\bcbar{9.9}
		%## } ##%
    \end{bchart}


\begin{center}
\small
%## var columns = o.Headers.Length; ##%
%## var first = true; ##%
\begin{longtable}{\cs{\verb$## print EnumerableExtension.Implode(Enumerable.Repeat("c", columns), "|"); ##$}{c|c|c}}
	\cs{
	\begin{comment}##
	foreach (var h in o.Headers) {
		print (first ? "" : @" & ") + @"\textbf{" + Tex.Escape(h) + @"}";
		first = false;
	}
	##\end{comment}
	}{\textbf{column1} & \textbf{column2} & \textbf{column3}}\\
	\hline
	\endhead
	\cs{
	\begin{comment}##
	foreach (var e in o.Entries) {
		first = true;
		foreach (var v in e) {
			print (first ? "" : @" & ") + Tex.Escape(v);
			first = false;
		}
		print @"\\\hline
	";
	}
	##\end{comment}
	}{a & b & c\\\hline
	d & e & f\\\hline}
\end{longtable}
\end{center}

\end{document}